\documentclass[../main.tex]{subfiles}
 
\begin{document}

\textbf{DTU01418, Introduction to Partial Differential Equations, Fall 2018}

\subsection{Problem 1: Polluted Water in Pipe}

\subsubsection{1.1}

It's assumed that the flux function in macroscopic model of pollution in pipe line is expressed as:
\begin{equation} \label{eq:A1_1}
    q(c) = v(c) \, c
\end{equation}

By focusing on the accumulated concentration from $(x_1,x_2)$ part of the pipe, we can derive the following equations:
\begin{align} 
    N(t) &= \int_{x_1}^{x_2} c(x,t) \, \mathrm{d}x \label{eq:A1_2} \\
    N`(t) &= \frac{d}{dt} \, \int_{x_1}^{x_2} c(x,t) \, \mathrm{d}x = \int_{x_1}^{x_2} c_t(x,t) \,\mathrm{d}x \label{eq:A1_3} \\
    N`(t) &= q(c(x_1,t)) - q(c(x_2,t)) = - \int_{x_1}^{x_2} q(c)_x(x,t) \, \mathrm{d}x \label{eq:A1_4}
\end{align}

we can get \textbf{conservation law for three variables} by eq.\ref{eq:A1_3} minus eq.\ref{eq:A1_4}:
\begin{align}
    \int_{x_1}^{x_2} [c_t(x,t) + q(c)_x(x,t)] \, \mathrm{d}x &= 0  \label{eq:A1_5} \\
    c_t(x,t) + [v(c) \, c]_x & =0 \label{eq:A1_6}
\end{align}

Since $v(c) = 0.5$, which is independent of $c$. We can write the initial value problem:
\begin{align} \label{eq:A1_7}
    \begin{cases}
        \, c_t + 0.5 \, c_x = 0 \\
        \, c(x,0) = g(x) = 1 + \cos{x} \, \sin{x} \\
        \, $t \geqslant 0$ \text{,} $x \in \mathbb{R}$
    \end{cases}
\end{align}

But we assume that the profile of polluted water from factory is the same as that is already let out.

\subsubsection{1.2}

Assume there is line expressing the relation between $t$ and $x$ on which the concentration are equal to the intersection $(x_0,0)$, and that means:
\begin{align}
    c(x(t),t) &= c(x_0,t) = g(x_0) \label{eq:A1_8} \\
    \frac{\mathrm{d}}{\mathrm{d} \, t} \, c(x(t),t) &= c_x \, x_t + c_t = 0 \label{eq:A1_9}
\end{align}

We can get the following equation by eq.\ref{eq:A1_9} minus eq.\ref{eq:A1_7}:
\begin{align}
    c_x \, (x_t - 0.5) &= 0  \label{eq:A1_10} \\
    \begin{cases} \label{eq:A1_11}
        \, c_x &= 0 \\
        \, x_t &= 0.5 \text{ , } x = 0.5 \, t + x_0
    \end{cases}
\end{align}

From eq.\ref{eq:A1_11}, we get:
\begin{equation}
    x_0 = x - 0.5 \, t \label{eq:A1_12}
\end{equation}

Substitute eq.\ref{eq:A1_12} into eq.\ref{eq:A1_8} to get:
\begin{equation}
    \begin{split}
        c(x(t),t) &= c(x_0,0) = g(x_0) \\
        &= g(x - 0.5 \, t) = 1 + \sin{(x - 0.5 \, t)} \, \cos{(x - 0.5 \, t)} \label{eq:A1_13}
    \end{split}
\end{equation}

Actually, we can only get, but we admit the eq.\ref{eq:A1_13} because the last assumption in the above section:
\begin{equation}
    c(x(t),t) = 1 + \sin{(x - 0.5 \, t)} \, \cos{(x - 0.5 \, t)} \text{ , } x \geqslant 0.5 \, t \label{eq:A1_14}
\end{equation}

\subsubsection{1.3}

When $t = 10$, the concentration yields:
\begin{equation}
    c(x,10) = 1 + 0.5 \, \sin{(2 \, (x - 5))} \label{eq:A1_15}
\end{equation}

This means the maximum value is 1.5 where $x = 5 + \frac{\pi}{4}$.

\subsubsection{1.4}

let's say when $t = t_1$, the polluted water starts flowing back toward the factory. We change the initial value problem of PDE to:
\begin{align} \label{eq:A1_16}
    \begin{cases}
        \, c_t - 0.5 \, c_x = 0 \\
        \, c(x,0) = 1 + 0.5 \, \sin{(2 \, (x - 0.5 \, t_1))} \\
        \, $t \geqslant 0$ \text{,} $x \in \mathbb{R}$
    \end{cases}
\end{align}

\subsection{Problem 2: Solve the Initial Value Problem}

\begin{align} \label{eq:A1_17}
    \begin{cases}
        \, u_t - 3/4 \, u_x = x^2 t \text{ ,  } x \in \mathbb{R} \text{ ,  } t \geqslant 0\\
        \, u(x,0) = g(x) = x \text{ ,  } x \in \mathbb{R} \text{ ,  }
    \end{cases}
\end{align}

This is a linear transport equation with distributed source and initial value problem.

The $ N`(t)$ can be expressed in two ways, one of which is the integration of $u_t$ in eq.\ref{eq:A1_17}:
\begin{align}
    N`(t) &= \frac{\mathrm{d}}{\mathrm{d} \, t} \, \int_{x_1}^{x_2} \, u \, \mathrm{d} x = \int_{x_1}^{x_2} \, u_t \, \mathrm{d} x = \int_{x_1}^{x_2} \, \left(3/4 \, u_x + x^2 \right) \, \mathrm{d} x \label{eq:A1_18} \\
    \begin{split} \label{eq:A1_19}
        N`(t) &= q[u(x_1,t)] - q[u(x_2,t)] + \int_{x_1}^{x_2} \, f(x,t) \, \mathrm{d} x \\
        &= - \int_{x_1}^{x_2} \, q(u)_x \, \mathrm{d} x + \int_{x_1}^{x_2} \, f(x,t) \, \mathrm{d} x
    \end{split}
\end{align}

\\
...... \\

According to proposition 4.1, the unique solution of the initial value problem is:
\begin{equation}
    u(x,t) &= g(x - v \, t) + \int_0^t \, f(x - v \, (t - s),s) \, \mathrm{d} s \label{eq:A1_20}
\end{equation}

From eq.\ref{eq:A1_17}, we can get:
\begin{align}
    v &= - \frac{3}{4} \\
    f(x) &= x^2 \\
    g(x) &= x
\end{align}

Then, eq.\ref{A1_20} can be derived:
\begin{align}
    u(x,t) &= (x + 3/4 \, t) + \int_0^t \, [x + 3/4 \, (t - s)]^2 \, \mathrm{d} s \\
    &= ......
\end{align}

\textcolor{red}{Remember to check the correctness of the solution, by substituting into the problem.}

\subsection{Burger's Equation}

\subsubsection{3.1}

\begin{align}
    \begin{cases} \label{eq:A1_26}
    u_t + u \, u_x = 0 \\
    u(x,0) = g(x) = \begin{cases}
                        - \, 1/3 & \text{ , } x < 1 \\
                        2 & \text{ , } x > 1
                    \end{cases}
    \end{cases}
\end{align}

In a similar way as subsection 1.2, we get:
\begin{align}
    u_x(x(t),t) \, [u(x(t),t) - x_t(t)] = 0 \label{eq:A1_27}
\end{align}

So,
\begin{align}
    x_t(t) &= u(x(t),t) = g(x_0) \label{eq:A1_28} \\
    x(t) &= g(x_0) \, t + x_0 \label{eq:A1_29} \\
    &= \begin{cases} \label{eq:A1_30}
           x_0 - \, 1/3 \,t & \text{ , } x_0 < 1 \\
           x_0 + 2 \, t & \text{ , } x_0 > 1
       \end{cases}
\end{align}

\begin{align}
    x_0 &= x - g(x_0) \, t \nonumber \\
        &= \begin{cases} \label{eq:A1_31}
              x + 1/3 \, t & \text{ , } x < 1 + 1/3 \, t \\
              x - 2 \, t & \text{ , } x > 1 - 2 \, t
          \end{cases}
\end{align}

\begin{align} \label{eq:A1_32}
    \begin{split}
        u(x(t),t) &= u(x_0,0) = g(x_0) \\
        &= \begin{cases}
               - \, 1/3 & \text{ when } x < 1 + 1/3 \, t \\
               2 & \text{ when } x > 1 - 2 \, t
           \end{cases}
    \end{split}
\end{align}

\subsubsection{1.3}

\begin{align} \label{eq:A1_33}
    \begin{cases} \label{eq:A1_26}
    u_t^{\epsilon} + u^{\epsilon} \, u_x^{\epsilon} = 0 \\
    u^{\epsilon}(x,0) = g^{\epsilon}(x) = \begin{cases}
                                   \, - \, 1/3 & \text{ , } x < 1 \\
                                   \, \frac{7 \, x - \epsilon - 7}{3 \, \epsilon}& \text{ , } 1 \leqslant x \leqslant 1 + \epsilon \\
                                   \, 2 & \text{ , } x > 1 + \epsilon
                                          \end{cases}
    \end{cases}
\end{align}

Similar as eq.\ref{eq:A1_29}:
\begin{align}
    x(t) &= g(x_0) \, t + x_0 \nonumber \\
    &= \begin{cases} \label{eq:A1_34}
           \, x_0 - \, 1/3 \,t & \text{ when } x_0 < 1 \\
           \, (\frac{7 \, t}{3 \, \epsilon} + 1) \, x_0 - \frac{(\epsilon + 7) \, t}{3 \, \epsilon}  & \text{ when } 1 \leqslant x_0 \leqslant 1 + \epsilon \\
           \, x_0 + 2 \, t & \text{ when } x_0 > 1
       \end{cases}
\end{align}

\begin{align}
    x_0 &= x - g(x_0) \, t \nonumber \\
        &= \begin{cases}
              \, x + 1/3 \, t & \text{ when } x < 1 + 1/3 \, t \\
              \, \frac{3 \, \epsilon}{7 \, t + 3 \, \epsilon} \, x + \frac{(7 + \epsilon) \, t}{7 \, t + 3 \, \epsilon} & \text{ when } 1 - \frac{t}{3} \leqslant x \leqslant 2 \, t + \epsilon + 1 \\
              \, x - 2 \, t & \text{ when } x > 1 - 2 \, t
          \end{cases}
\end{align}



\end{document}