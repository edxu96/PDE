\documentclass[../main.tex]{subfiles}
 
\begin{document}

\subsection{Opportunities and Dangers of Finite Difference}

Assume a function $u(x)$. Choose a mesh size $\Delta x$, and approximate the value $u(j \, \Delta x)$ for $x = j \, \Delta x$ (divide $x$ to $j$ parts of $\Delta x$) by a number $u_j$ indexed by an integer j:
\begin{align*}
    & u_j \sim u(j \, \Delta x) \sim u(x)\\
    \text{Similarly, } & u_{j+1} \sim u((j+1) \, \Delta x) \sim u(x + \Delta x)
\end{align*}

Assume $u(j \, \Delta x) \in C^2(I)$, then the Taylor expansion of $u(x + \Delta x)$ and $u(x - \Delta x)$ are:
\begin{align*}
    u(x + \Delta x) = u(x) + u`(x) \, \Delta x + O \, (\Delta x)^2 \\
    u(x - \Delta x) = u(x) - u`(x) \, \Delta x + O \, (\Delta x)^2
\end{align*}

Then the three standard approximation for the first derivative $\frac{\partial u}{\partial x} \, (j \, \Delta x)$ are shown. Pay attention to the change of the order of omission term $O \, (\Delta x)$, which causes truncation error mentioned below and has an impact on the final outcome.
\begin{align}
    & \text{The backward differences: } u`(j \, \Delta x) \approx \frac{u_j - u_{j-1}}{\Delta x} + \xcancel{O(\Delta x)} \label{eq:FD1} \\
    & \text{The forward differences: } u`(j \, \Delta x) \approx \frac{u_{j+1} - u_j}{\Delta x} + \xcancel{O(\Delta x)} \label{eq:FD2} \\
    & \text{The centered differences: } u`(j \, \Delta x) \approx \frac{u_{j+1} - u_{j-1}}{2 \, \Delta x} + \xcancel{O(\Delta x)} \label{eq:FD3} 
\end{align}

Assume $u(j \, \Delta x) \in C^3(I)$, then the Taylor expansion can be more detailed:
\begin{align*}
    u(x + \Delta x) = u(x) + u`(x) \, \Delta x + \frac{1}{2} \, u``(x) \, (\Delta x)^2 + O \, (\Delta x)^3
\end{align*}

For the second derivative, the simplest approximation is the centered second difference:
\begin{align}
    u``(j \, \Delta x) \approx \frac{u_{j+1} - 2\, u_j + u_{j-1}}{(\Delta x)^2} + \xcancel{O(\Delta x)^2} \label{eq:FD4}
\end{align}

For functions of two variables $u(x,t)$, we choose a mesh size for two variables. We write:
\begin{align*}
    u_j^n \sim u(j \, \Delta x, n \, \Delta t) \sim u(x,t)
\end{align*}

And we can write similar expressions like equation (\ref{eq:FD1}) - (\ref{eq:FD4}).

Two kinds of errors can be introduced in a computation using such approximation:
\begin{itemize}
  \item \textbf{Truncation Error}: The error in the solutions by the approximation itself.
  \item \textbf{Round-off Error}: The limitation that only a certain number of digits, typically 8 or 16, are obtained by the computer at each step of the computation.
\end{itemize}






\subsection{Approximations of Diffusion}

The choice of the mesh $\Delta t$ relative to the mesh $\Delta x$ is called Courant Number:
\begin{align}
    s = \frac{\Delta t}{(\Delta x)^2} \label{eq:FD5}
\end{align}

Let's see a simple problem:
\begin{align}
    u_t - u_{xx} = 0 \text{ , } u(x,0) = \phi(x) \label{eq:FD6}
\end{align}

The finite difference equation of eq.\ref{eq:FD6} can be write with local truncation error:
\begin{align}
    \frac{u_j^{n+1} - u_j^{n}}{\Delta t} + \xcancel{O(\Delta t)} & = \frac{u_{j+1}^n - 2 \, u_j^n + u_{j-1}^n}{(\Delta x)^2} + \xcancel{O(\Delta x)^2} \label{eq:FD7}
\end{align}

Substitute eq.\ref{eq:FD5}, we can get:
\begin{align}
    u_j^{(n+1)} = s \, (u_{j+1}^n + u_{j-1}^n) + (1 - 2 \, s) \, u_j^n \label{eq:FD8}
\end{align}

And the following diagram can be a visual explanation of eq.\ref{eq:FD8}.
\begin{center}
\begin{tabular}{|P{3em} P{3em} P{3em} P{3em} P{3em} P{3em} P{3em}}
    t     &     &       &        &       & &   \\ 
          &     &       &        &       & &   \\
    (n+1) &     &       & *      &       & &   \\
    n     &     & s     & (1-2s) & s     & &   \\
    ...   &     &       &        &       & &   \\
    1     &     &       &        &       & &   \\
    0     & ... & (j-1) & j      & (j+1) & & x \\
    \hline
\end{tabular}
\end{center}

So, from initial value $u(x,0) = \phi(x)$ in eq.\ref{eq:FD5}, we can get any $u(x,j)$. There may be need of other necessary additional boundary condition when come down to the value along the boundary.

But there is a large difference in the outcome of the above method because of chosen value of $s$, This is called the stability criterion which is discussed below.

\subsubsection{Stability Criterion}

To actually demonstrate that this is the stability condition, we separate the variables in the difference equation. Thus we look for solutions of equation \ref{eq:FD8} of the form:
\begin{align}
    \begin{split} \label{eq:FD18}
        u_j^n &=  X_j \, T_n \\
              &= \frac{u_j^{n+1} - s \, (u_{j+1}^n + u_{j-1}^n)}{(1 - 2 \, s)}
    \end{split} 
\end{align}

Thus,
\begin{align}
    \begin{split} \label{eq:FD19}
    \frac{T_{n+1}}{T_n} &= \frac{u_j^{n+1}}{u_j^n} \\
                        &= s \, \frac{X_{j+1} + X_{j-1}}{X_j} + 1 - 2 \, s
    \end{split}
\end{align}

Both sides of equation \ref{eq:FD18} must be a constant $\xi$ independent of $j$ and $n$. Therefore,
\begin{align}
    T_n &= \xi^n \, T_0 \label{eq:FD20} \\
    \frac{T_{n+1}}{T_n} &= \xi \nonumber
\end{align}

So,
\begin{align}
    s \, \frac{X_{j+1} + X_{j-1}}{X_j} + 1 - 2 \, s  = \xi \label{eq:FD21}
\end{align}

To solve the spatial eq.\ref{eq:FD21}, we argue that it is a discrete version of a second-order ODE which has sine and cosine solutions. Therefore, we guess solutions of eq.\ref{eq:FD21} of the form:
\begin{align}
    X_j = A \, \cos{(j \, \theta)} + B \, \sin{(j \, \theta)} \label{eq:FD22}
\end{align}
for some $\theta$, where $A$ and $B$ are arbitrary. The boundary condition $X_0 = 0$ at $j \, \theta= 0$ implies that $A = 0$. So we can freely set $B = 1$. Then $X_j = \sin{(j \, \theta)}$.

Furthermore, the boundary condition $X_J = 0$ at $j = J$ implies that $\sin{(j \, \theta)} = 0$. Thus $J \, \theta = k \, \pi$ for some integer $k$. But the discretization into $J$ equal intervals of length $\Delta x$ meas that $J = \pi / \Delta x$. Therefore, $\theta = k \, \Delta x$ and
\begin{align}
    X_j = \sin{(j \, k \, \Delta x)} \label{eq:FD23}
\end{align}

Now eq.\ref{eq:FD21} takes the form:
\begin{align*}
    s \, \frac{\sin{((j+1) \, k \, \Delta x)} + \sin{((j-1) \, k \, \Delta x)}}{\sin{(j \, k \, \Delta x)}} + 1 - 2 \, s  = \xi \nonumber
\end{align*}
or
\begin{align}
    \xi = \xi(k) = 1 - 2 \, s \, [1 - \cos{(k \, \Delta x)}] \label{eq:FD24}
\end{align}

According to eq.\ref{eq:FD20}, the growth in time $t = n \, \Delta t$ at the wave number k is governed by the powers $\xi(k)^n$. So, \textbf{unless $|\xi(k)| \leqslant 1$ for all $k$ the scheme is unstable} and could not possibly approximate the true (exact) solution. (Recall that the true solution tends to zero as $t \rightarrow \inf$.) Now we analyze eq.\ref{eq:FD24} to determine whether $|\xi(k)| \leqslant 1$ or not. Since factor $1 - \cos{(k \, \Delta x)}$ ranges between 0 and 2, we have $1 - 4 \, s \leqslant \xi(k) \leqslant 1$. So stability requires that $1 - 4 \, s \geqslant - 1$, which means that
\begin{align}
    \frac{\Delta t}{(\Delta x)^2} = s \leqslant \frac{1}{2} \label{eq:FD25}
\end{align}





\end{document}